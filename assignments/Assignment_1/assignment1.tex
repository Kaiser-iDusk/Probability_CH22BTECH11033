\documentclass[journal,12pt,two column]{IEEEtran}
%\usepackage{setspace}
\usepackage{listings}
\usepackage{amssymb}
\usepackage[cmex10]{amsmath}
\usepackage{amsthm}
\usepackage[export]{adjustbox}
\usepackage{bm}
\def\inputGnumericTable{} 

\usepackage[latin1]{inputenc}                                 
\usepackage{color}                                            
\usepackage{array} 
\usepackage{longtable} 
\usepackage{calc}                                             
\usepackage{multirow}                                         
\usepackage{hhline}                                           
\usepackage{ifthen}  
\usepackage{mathtools}
\usepackage{tikz}
\usepackage{listings}
\usepackage{color}                                            %%
\usepackage{array}                                            %%
\usepackage{caption} 
\usepackage{graphicx}

\title{A1110 Assignment 1 \\ 11.16.4.9}
\author{SHIRSENDU PAL \\ CH22BTECH11033 \\
        \thanks{*The student is from Department of Chemical Engineering, Indian Institute of Technology, Hydeabad. e-mail: ch22btech11033@iith.ac.in}}
\providecommand{\pr}[1]{\ensuremath{\Pr\left(#1\right)}}
\providecommand{\qfunc}[1]{\ensuremath{Q\left(#1\right)}}
\providecommand{\sbrak}[1]{\ensuremath{{}\left[#1\right]}}
\providecommand{\lsbrak}[1]{\ensuremath{{}\left[#1\right.}}
\providecommand{\rsbrak}[1]{\ensuremath{{}\left.#1\right]}}
\providecommand{\brak}[1]{\ensuremath{\left(#1\right)}}
\providecommand{\lbrak}[1]{\ensuremath{\left(#1\right.}}
\providecommand{\rbrak}[1]{\ensuremath{\left.#1\right)}}
\providecommand{\cbrak}[1]{\ensuremath{\left\{#1\right\}}}
\providecommand{\lcbrak}[1]{\ensuremath{\left\{#1\right.}}
\providecommand{\rcbrak}[1]{\ensuremath{\left.#1\right\}}}
\newcommand*{\permcomb}[4][0mu]{{{}^{#3}\mkern#1#2_{#4}}}
\newcommand*{\perm}[1][-3mu]{\permcomb[#1]{P}}
\newcommand*{\comb}[1][-1mu]{\permcomb[#1]{C}}
\renewcommand{\thetable}{\arabic{table}} 
\newcommand{\question}{\noindent \textbf{Question: }}	
\newcommand{\solution}{\noindent \textbf{Solution: }}
\begin{document}
\maketitle
\question: 16 : If 4-digit numbers greater than 5,000 are randomly formed from the digits 0, 1, 3, 5, and 7, what is the probability of forming a number divisible by 5 when, (i) the digits are repeated? (ii) the repetition of digits is not allowed?\\
\solution : Let;
 \begin{align*} 
    R \coloneqq 
        \begin{cases}
            1, & \text{if repetition allowed}\\
            0, & \text{if repetition not allowed}\\
        \end{cases}
    \\ \\
    T \coloneqq \{0, 1, 3, 5, 7\}\\ \\
    S \coloneqq \{x \mid (x \geq 5001) \land (\operatorname{digits}(x) \in T)\}
 \end{align*}
 \begin{align*}
 	\therefore \mid S \mid = 2*5^3 -1 = 249 \\  
 \end{align*}
 $\because$ 2 choices for first position 2 choices for last position and 5 choices for middle position. Also, exclude case for 5000.\\
 $(i)$ Let:
  \begin{align*}
  	E_1 \coloneqq \{x \mid (x \in S)\land(5 \mid x)\land(R = 1)\}\\
        \therefore \mid E_1\mid = 2 * 2 * 5^2 -1 = 99.
  \end{align*}
  \begin{equation*}
    \pr{E_1} = \frac{\mid E_1 \mid}{\mid S \mid} = \frac{33}{83}.
  \end{equation*}
$(ii)$ Let:
    \begin{align*}
        E_2 \coloneqq \{x \mid (x \in S)\land(5 \mid x)\land (R=0)\}\\
        \therefore \mid E_2 \mid = 1*1*^3P_2 + 1*2*^3P_2 = 18
    \end{align*}
    \begin{equation*}
	    \pr{E_2} = \frac{\mid E_2 \mid}{\mid S \mid} = \frac{6}{83}.
    \end{equation*}
\end{document}
