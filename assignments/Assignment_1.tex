\documentclass[12pt, a4letter]{article}
\usepackage{amsmath}
\usepackage{mathtools}
\usepackage{graphicx} % Required for inserting images

\title{\textbf{AI-1110 Assignment 1}}
\author{\textbf{Shirsendu Pal} \textbf{CH22BTECH11033}}
\date{April 2023}

\begin{document}

\maketitle{}

\section*{Problem - \textbf{11.16.4.9}}
\begin{enumerate}
    \item \textbf{Question:} If 4-digit numbers greater than 5,000 are randomly formed from the digits 0, 1, 3, 5, and 7, what is the probability of forming a number divisible by 5 when, (i) the digits are repeated? (ii) the repetition of digits is not allowed?
    
    \textbf{Solution:} Let us define the digits set \textbf{T} and sample space \textbf{S} such that
    \begin{align*}
        T &\coloneqq \{0, 1, 3, 5, 7 \}
        \intertext{and}
        S &\coloneqq \{x \mid (x \geq 5001) \land (\operatorname{digits}(x) \in T) \}
    \end{align*}
    Thus, $\mid S \mid = 2 * 5 ^ 3 - 1$ because first digit-place(i,e. leftmost digit) of $x \in S$ can have digits from $\{5, 7\}$. So, we have $2$ choices. The next $3$ places can be filled in $\mid T \mid = 5$ ways. Finally, we exclude the case of $5000$. So, in total we have $2*5^3 -1$ elements in $S$.

    \textbf{NOTE: } A number is divisible by 5 $\iff$ its last digit is 0 or 5.

     \textbf{(i)}  Let $E_1$ denote the event where the numbers $\in S$ and repetition of digits from $T$ is allowed. Then we can pick 2 digits $\{5, 7\}$ as the left-most digits, while we can have $\{0, 5\}$ as the right-most digits. Whereas, the middle two digit places can be filled in 5 ways. We also exclude the case where the number formed is $5000$. So, we have a total of $2*2*5^2 - 1$ ways. So, 
    \begin{equation*}
        \mathcal{P}(E_1) = \frac{2 * 2 * 5^2 - 1}{2 * 5^3 - 1} = \frac{99}{249} = \frac{33}{83}.
    \end{equation*}

    \textbf{(ii)} Let $E_2$ denote the event where the numbers $\in S$ and repetition of digits from $T$ is \textbf{not} allowed.  Then we have $2$ possibilities for left-most position $\{5, 7\}$. Let us consider that the left-most position has $5$. It implies that right-most position is bound to be $0$ for $E_2$ to occur. The middle $2$ positions can be filled with digits from $T - \{0, 5\}$. That is, we can do it in $^3P_2$ ways. And if the left-most position has $7$, then right-most position has $2$ possibilities $\{0, 5\}$. The middle positions can then have $^3P_2$ possibilities of filling up. Hence, total number of ways is $(1*^3P_2*1) + (1*^3P_2*2)$. So,
    \begin{equation*}
        \mathcal{P}(E_2) = \frac{(1*^3P_2*1) + (1*^3P_2*2)}{2 * 5^3 - 1} = \frac{6+12}{249} = \frac{6}{83}.
    \end{equation*}

    \begin{align*}
        \texttt{\textbf{THE END}}
    \end{align*}
\end{enumerate}
\end{document}
